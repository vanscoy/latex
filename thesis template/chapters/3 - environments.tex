\chapter{Environments}

Environments in LaTeX are used to apply specific formatting to part of the document\footnote{\url{https://www.overleaf.com/learn/latex/Environments}}.


%%%%%%%%%%%%%%%%%%%%%%%%%%%%%%%%%%%%%%%%%%%%%%%%%%%%%%%%%%%%%%%%%%%%%%%%%%%%%%%%
\section{Theorems}

For Theorem environments, we use the \verb|amsthm| package. This allows us to define environments that are frequently used such as \verb|thm| for theorem, \verb|lem| for lemma, and so on. We can also use the \verb|thmtools| package to create boxed or shaded theorems for emphasis. Here are some examples.

\begin{theorem}[A theorem]
	\label{thm:big_result1}
	This is how we state a theorem.
\end{theorem}

\begin{theoremshaded}[Shaded]
	\label{thm:big_result2}
	For emphasis, we can put it in a shaded box.
\end{theoremshaded}

\begin{theoremboxed}[Outlined]
	\label{thm:big_result3}
	Another way to create emphasis is with an outlined box.
\end{theoremboxed}

\begin{proof}
We can write proofs using the \verb|proof| environment.
\end{proof}

The \verb|thmtools| package also provides \verb|restatable|, which is useful if you want to state the same result more than once (say, in the introduction and later in the paper), but don't want to give it a new label and equation numbers. See the documentation for more details\footnote{\url{https://ctan.math.illinois.edu/macros/latex/contrib/thmtools/doc/thmtools-manual.pdf}}.


%%%%%%%%%%%%%%%%%%%%%%%%%%%%%%%%%%%%%%%%%%%%%%%%%%%%%%%%%%%%%%%%%%%%%%%%%%%%%%%%
\section{Lists}

Create bulleted lists using the \verb|itemize| environment. For example:
\begin{itemize}
	\item First item
	\item Second item
	\item Third item
\end{itemize}
Numbered lists are created using the \verb|enumerate| environment. For customization, we use the \verb|enumitem| package with the \verb|shortlabels| option. This allows us to write customized lists easily. For example,
\bigskip

\hfill
\begin{minipage}{0.45\linewidth}
\begin{verbatim}
\begin{enumerate}[(i)]
    \item first item \label{x}
    \item second item \label{y}
    \item third item \label{z}
\end{enumerate}
\end{verbatim}
\end{minipage}
\hfill produces \hfill
\begin{minipage}{0.3\linewidth}
\begin{enumerate}[(i)]
	\item first item \label{x}
	\item second item \label{y}
 	\item third item \label{z}
\end{enumerate}
\end{minipage}
\bigskip

We can refer to items using \verb|\cref| as before. For example, the command \verb|\cref{x,y,z}| produces \cref{x,y,z}.