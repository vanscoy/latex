\chapter{Introduction to \LaTeX}

LaTeX is a typesetting program used to create high-quality documents\footnote{\url{https://www.latex-project.org/about/}}. It is free to use, and is based on the typesetting system TeX created by Donald Knuth in 1978. Nearly all professional papers are created using LaTeX, and any research papers submitted to IEEE should be written using it. While there is a bit of a learning curve to get going, you will find that it is a very elegant and powerful program once you have mastered it!

Standard word processors --- such as Microsoft Word --- are called \emph{what you see is what you get}, meaning that you directly edit the final document with all of its final formatting. In contrast, LaTeX separates the content from its layout. You write content in \texttt{.tex} files, which are then processed by LaTeX to produce the final document, typically a pdf. This frees you from the burden of thinking about the final layout of the document while you write. For example, this template is written using the custom \texttt{thesis.cls} style file which tells LaTeX how to format the document to meet the requirements of a thesis. But you do not need to know anything about the formatting to use the template. You just fill in your content and let LaTeX take care of building the final document for you. You are encouraged to look at the source code of this document for reference, since it was made using LaTeX.

To learn more about LaTeX, see some of the many references online\footnote{\url{https://www.overleaf.com/learn/latex/Learn_LaTeX_in_30_minutes}}.


%%%%%%%%%%%%%%%%%%%%%%%%%%%%%%%%%%%%%%%%%%%%%%%%%%%%%%%%%%%%%%%%%%%%%%%%%%%%%%%%
\section{Section title}

You can make sections using the \verb|\section| command. Sections are automatically added to the table of contents (but you need to run \texttt{pdflatex} twice for them to update!).

Put a blank line between text to put it in a separate paragraph.

\paragraph{Paragraph title}

The \verb|\paragraph| command can be used to break up long blocks of text and help the reader. Paragraph titles should have the first word capitalized and then the others lowercase, and they should end with a period.

\paragraph{Acronyms}

We should try to avoid acronyms wherever possible. The only acronyms that should be used are those that are so common that they are more easily recognized than their definitions, such as ``the \ac{KYP} lemma'' or \ac{TCP/IP}. As Richard Murray is fond of saying, defining an acronym just to save typing in a paper is very bad for a reader. We can read a familiar sequence of words almost as fast as we can read an acronym, and much faster if we have to pause to remember what the acronym is actually for. Generally speaking, we should limit ourselves to at most one or two new acronyms, and they should be reserved for things like algorithm names that we hope other people will also use. To organize acronyms, you can use the \texttt{acronym} package. For example, you can use the acronyms defined above as \ac{KYP} and \ac{TCP/IP}.


%%%%%%%%%%%%%%%%%%%%%%%%%%%%%%%%%%%%%%%%%%%%%%%%%%%%%%%%%%%%%%%%%%%%%%%%%%%%%%%%
\section{Compiling}

To build the final document, you need to compile the \texttt{.tex} files. You can do this by installing LaTeX on your computer using a distribution such as MikTex\footnote{\url{https://miktex.org/}}, or using the online LaTeX editor Overleaf\footnote{\url{https://www.overleaf.com/}}. If you choose to download LaTeX, a good editor is Visual Studio Code\footnote{\url{https://code.visualstudio.com/}}, which has a LaTeX extension\footnote{\url{https://marketplace.visualstudio.com/items?itemName=James-Yu.latex-workshop}} that provides code highlighting and easy compilation. You can also build the files from the command line using the following:
{\setlength{\jot}{-3pt}
  \begin{align*}
    &\texttt{pdflatex thesis.tex}\\
    &\texttt{bibtex thesis.aux}\\
    &\texttt{pdflatex thesis.tex}\\
    &\texttt{pdflatex thesis.tex}
  \end{align*}}
It may seem odd to have to rerun the command so many times, but each pass stores certain auxiliary information (if you look in the main folder after running these commands, there are many new files such as \texttt{.aux}, \texttt{.lof}, \texttt{.log}, \texttt{.lot}, \texttt{.out}, and \texttt{.toc}). These files contain the information required to make things like the table of contents, list of figures, list of tables, references, etc. The first command creates the auxiliary file \texttt{thesis.aux} that indicates which references are cited. The second command runs bibtex on the file \texttt{thesis.aux} to create the bibliography in \texttt{thesis.bbl}. The third command then builds the document with the bibliography, and the bibliography labels are written into the \texttt{.aux} file. And finally, the compiler has access to all the required information on the last run to build the final document correctly.

In a typical workflow, you do not need to rerun all of these commands every time you edit the document. Typically, you just need to rerun pdflatex to update the pdf. If there were errors, however, you will need to rerun bibtex to fix the bibliography. Make sure to run all four commands to build the final document that you submit!

These four commands are in the file \texttt{make.bat}, so you can also run that file to make the final document.

\subsection{This is a subsection}

You can organize your document using subsections.

\subsubsection{This is a subsubsection}

And subsubsections\ldots

\paragraph{This is a paragraph}

And even paragraphs. Note that paragraphs continue on the same line as the heading, and there is a period automatically inserted after the heading. By default, subsections, subsubsections, and paragraphs are not included in the table of contents.